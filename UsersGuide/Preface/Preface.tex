This documentation is a work in progress.  \pelelm\ is rapidly
evolving, so portions of the text are likely to be incomplete. \\

\noindent \pelelm\ is developed primarily at the Center for Computational Sciences and
Engineering at Lawrence Berkeley National Laboratory.  Development has been supported
by the SciDAC Program of the DOE Office of Mathematics, Information, and Computational Sciences
under the U.S. Department of Energy under contract No.\ DE-AC02-05CH11231. \\

Past and present contributors to the \pelelm\ code base include: % git shortlog -s -n
\begin{quote}
Ann Almgren\\
Andy Aspden\\
John Bell\\
Vince Beckner\\
Marc Day\\
Matthew Emmett\\
Candace Gilet\\
Ray Grout\\
Mike Lijewski\\
Andy Nonaka\\
Will Pazner\\
Chuck Rendleman\\
Weiqun Zhang\\
\end{quote}

\noindent \pelelm\ uses the \amrex\ library,
developed at the Center for Computational Sciences and Engineering (CCSE)
at Lawrence Berkeley National Laboratory.  \amrex\ is distributed
separately from \pelelm\ (see \url{https://github.com/AMReX-Codes/amrex})
\\

\noindent The current version of the \pelelm\ User's Guide can be found in 
the \pelelm\ git repository (on ORNL GitLab)
under {\tt PeleLM/Docs/UsersGuide/}.

\noindent For questions about obtaining/using the code or this 
User's Guide, contact Marc Day {\tt MSDay@lbl.gov} and Andy Nonaka 
{\tt AJNonaka@lbl.gov} of CCSE.

\noindent Mailing list details coming soon.

