Our equations look like:
\begin{equation}
\frac{\partial\Ub}{\partial t} = \nabla\cdot\Fb + \Sb_{\rm react} + \Sb,
\end{equation}
where $\Fb$ is the flux vector, $\Sb_{\rm react}$ are the reaction source terms, and $\Sb$ are the non-reaction source terms, which includes any user-defined external sources, $\Sb_{\rm ext}$.  We use Strang splitting to discretize the advection-reaction equations.  In summary, for each time step, we update the conservative variables, $\Ub$, by reacting for half a time step, advecting for a full time step (ignoring the reaction terms), and reacting for half a time step.  In summary,
\begin{equation}
\Ub^n = \Ub^n + \frac{\dt}{2}\Sb_{\rm react}^n,
\end{equation}
\begin{equation}
\Ub^{n+1} = \Ub^n - \Delta t \nabla \cdot\Fb^\nph + \dt\frac{\Sb^n + \Sb^{n+1}}{2},
\end{equation}
\begin{equation}
\Ub^{n+1} = \Ub^{n+1} + \frac{\dt}{2}\Sb_{\rm react}^{n+1},
\end{equation}
The construction of $F$ is purely explicit, and based on an unsplit second-order Godunov method.  We predict the standard primitive variables, as well as $\rho e$, at time-centered edges and use an approximate Riemann solver construct fluxes.  At the beginning of the time step, we assume that $\Ub$ and $\phi$ are defined consistently, i.e., $\rho^n$ and $\phi^n$ satisfy equation (\ref{eq:Self Gravity}).\\

CASTRO also supports radiation (Chapter \ref{Chap:Radiation}) and level sets (Chapter \ref{Chap:Level Sets}).  We omit the details in this section.  Here is the single-level algorithm:
\begin{description}
\item[Step 1:] {\em React $\Delta t/2$.}

Update the solution due to the effect of reactions over half a time step.
\begin{eqnarray}
(\rho E)^n &=& (\rho E)^n - \frac{\dt}{2}\sum_k(\rho q_k\omegadot_k)^n,\\
(\rho X_k)^n &=& (\rho X_k)^n + \frac{\dt}{2}(\rho\omegadot_k)^n.
\end{eqnarray}
\item[Step 2:] {\em Solve for gravity.}

Solve for gravity using:
\begin{equation}
\gb^n = \nabla\phi^n, \qquad 
\Delta\phi^n = -4\pi G\rho^n,
\end{equation}
or use one of the simpler gravity types.

\item[Step 3:] {\em Compute explicit source terms.}

We now compute explicit source terms for each variable in $\Qb$ and $\Ub$.  The primitive variable source terms will be used to construct time-centered fluxes.  The conserved variable source will be used to advance the solution.  We neglect reaction source terms since they are accounted for in {\bf Steps 1} and {\bf 6}.  The source terms are:
\begin{equation}
\Sb_{\Qb}^n =
\left(\begin{array}{c}
S_\rho \\
\Sb_{\ub} \\
S_p \\
S_{\rho e} \\
S_{A_k} \\
S_{X_k} \\
S_{Y_k}
\end{array}\right)^n
=
\left(\begin{array}{c}
S_{{\rm ext},\rho} \\
\gb + \frac{1}{\rho}\Sb_{{\rm ext},\rho\ub} \\
\frac{1}{\rho}\frac{\partial p}{\partial e}S_{{\rm ext},\rho E} + \frac{\partial p}{\partial\rho}S_{{\rm ext}\rho} \\
\nabla\cdot\kappa\nabla T + S_{{\rm ext},\rho E} \\
\frac{1}{\rho}S_{{\rm ext},\rho A_k} \\
\frac{1}{\rho}S_{{\rm ext},\rho X_k} \\
\frac{1}{\rho}S_{{\rm ext},\rho Y_k}
\end{array}\right)^n,
\end{equation}
\begin{equation}
\Sb_{\Ub}^n =
\left(\begin{array}{c}
\Sb_{\rho\ub} \\
S_{\rho E} \\
S_{\rho A_k} \\
S_{\rho X_k} \\
S_{\rho Y_k}
\end{array}\right)^n
=
\left(\begin{array}{c}
\rho \gb + \Sb_{{\rm ext},\rho\ub} \\
\rho \ub \cdot \gb + \nabla\cdot\kappa\nabla T + S_{{\rm ext},\rho E} \\
S_{{\rm ext},\rho A_k} \\
S_{{\rm ext},\rho X_k} \\
S_{{\rm ext},\rho Y_k}
\end{array}\right)^n.
\end{equation}

\item[Step 3:] {\em Advect $\Delta t$.}

The goal is to advance
\begin{equation}
\Ub^{n+1} = \Ub^n - \dt\nabla\cdot\Fb^\nph + \dt\Sb^n.
\end{equation}
neglecting reaction terms.  Note that since the source term is not time centered, this is not a second-order method.  After the advective update, we correct the solution, effectively time-centering the source term. The advection step is complicated, and more detail is given in Section \ref{Sec:Advection Step}.  Here is the summarized version:
\begin{enumerate}
\item Compute primitive variables.
\item Predict primitive variables to time-centered edges.
\item Solve the Riemann problem.
\item Compute fluxes and update.
\end{enumerate}
\item[Step 4:] {\em Solve for updated gravity.}

Solve for gravity using:
\begin{equation}
\gb^{n+1} = \nabla\phi^{n+1}; \qquad \Delta\phi^{n+1} = -4\pi G\rho^{n+1},
\end{equation}
or use one of the simpler gravity types.
\item[Step 5:] {\em Correct solution with time-centered source terms.}

We need to correct the solution by effectively time-centering the source terms.  These corrections are to be performed sequentially since new source term evaluations may depend on previous corrections.

First, we correct the solution with the updated gravity:
\begin{eqnarray}
(\rho\ub)^{n+1} &=& (\rho\ub)^{n+1} + \frac{\dt}{2}\left[(\rho\gb)^{n+1} - (\rho\gb)^n\right], \\
(\rho E)^{n+1} &=& (\rho E)^{n+1} + \frac{\dt}{2}\left[\left(\rho\ub\cdot\gb\right)^{n+1} - \left(\rho\ub\cdot\gb\right)^n\right].
\end{eqnarray}

Next, we correct $\Ub$ with updated external sources.  For example, for the momentum, we correct using
\begin{equation}
(\rho\ub)^{n+1} = (\rho\ub)^{n+1} + \frac{\dt}{2}\left(\Sb_{{\rm ext},\rho\ub}^{n+1} - \Sb_{{\rm ext},\rho\ub}^n\right).
\end{equation}
We correct $\rho E, \rho A_k, \rho X_k$, and $\rho Y_k$ in an analogous manner.

Finally, we correct the solution with updated thermal diffusion using
\begin{equation}
(\rho E)^{n+1} = (\rho E)^{n+1} + \frac{\dt}{2}\left(\nabla\cdot\kappa\nabla T^{n+1} - \nabla\cdot\kappa\nabla T^n\right).
\end{equation}
\item[Step 6:] {\em React $\Delta t/2$.}

Update the solution due to the effect of reactions over half a time step.
\begin{eqnarray}
(\rho E)^{n+1} &=& (\rho E)^{n+1} - \frac{\dt}{2}\sum_k(\rho q_k\omegadot_k)^{n+1},\\
(\rho X_k)^{n+1} &=& (\rho X_k)^{n+1} + \frac{\dt}{2}(\rho\omegadot_k)^{n+1}.
\end{eqnarray}
\item[Step 7:] {\em Modify auxiliary variables.}

This is problem-dependent.  By default we treat the auxiliary variables as 
advected quantities, so no additional steps are required.
\end{description}
Thus concludes the single-level algorithm description.

\subsection{Castro::advance()}

\noindent

if (doReact)

\hspace{.1in}  strangChem()

end if

if (doGrav)

\hspace{.1in}  define oldGravityVector

end if

if (Diffusion)

\hspace{.1in}  getOldDiffusionTerm()

end if

if (addExtSource)

\hspace{.1in}  getSource() at old time

end if

AdvanceSolution()

if (doGrav)

\hspace{.1in}  define newGravityVector

\hspace{.1in}  correct solution due to new gravity

end if

if (addExtSource)

\hspace{.1in}  getSource() at new time

\hspace{.1in}  correct solution due to new source

end if

if (Diffusion)

\hspace{.1in}  getNewDiffusionTerm()

\hspace{.1in}  correct solution due to new diffusion term

\hspace{.1in}  computeTemp()

end if

if (doReact)

\hspace{.1in}  strangChem()

end if

if (advanceAux)

\hspace{.1in}  advanceAux()

end if

if (LevelSet)

\hspace{.1in}  advanceLevelSet()

end if
\section{Advection Step}\label{Sec:Advection Step}
There are four major steps in the advective update, detailed below.
\subsection{Compute Primitive Variables}\label{Sec:Compute Primitive Variables}
We compute the primtive variables from the conserved variables.
\begin{itemize}
\item $\rho, \rho e$ - directly copy these from the conserved state vector
\item $\ub, A_k, X_k, Y_k$ - copy these from the conserved state vector, dividing by $\rho$
\item $p,T$ - use the EOS.  First, if {\bf castro.allow\_negative\_energy} = 0 (it defaults to 1) and $e < 0$, we do the following:
\begin{enumerate}
\item Use the EOS to set $e = e(\rho,T_{\rm small},X_k)$.
\item If $e < 0$, abort the program with an error message.
\end{enumerate}
Now, use the EOS to compute $p,T = p,T(\rho,e,X_k)$.
\end{itemize}
We also compute the flattening coefficient, $\chi\in[0,1]$, used in the edge state prediction to further limit slopes near strong shocks.  We use the same flattening procedure described in the the FLASH paper.  A flattening coefficient of 1 indicates that no additional limiting takes place; a flattening coefficient of 0 means we effectively drop order to a first-order Godunov scheme (this convention is opposite of that used in the FLASH paper).  For each cell, we compute the flattening coefficient for each spatial direction, and choose the minimum value over all directions.  As an example, to compute the flattening for the x-direction, here are the steps:
\begin{enumerate}
\item Define $\zeta$
\begin{equation}
\zeta_i = \frac{p_{i+1}-p_{i-1}}{\max\left(p_{\rm small},|p_{i+2}-p_{i-2}|\right)}.
\end{equation}
\item Define $\tilde\chi$
\begin{equation}
\tilde\chi_i = \min\left\{1,\max[0,a(\zeta_i - b)]\right\},
\end{equation}
where $a=10$ and $b=0.75$ are tunable parameters.  We are essentially setting $\tilde\chi_i=a(\zeta_i-b)$, and then constraining $\tilde\chi_i$ to lie in the range $[0,1]$.  Then, if either $u_{i+1}-u_{i-1}<0$ or
\begin{equation}
\frac{p_{i+1}-p_{i-1}}{\min(p_{i+1},p_{i-1})} \le c,
\end{equation}
where $c=1/3$ is a tunable parameter, then set $\tilde\chi_i=0$.
\item Define $\chi$
\begin{equation}
\chi_i =
\begin{cases}
1 - \max(\tilde\chi_i,\tilde\chi_{i-1}) & p_{i+1}-p_{i-1} > 0 \\
1 - \max(\tilde\chi_i,\tilde\chi_{i+1}) & \text{otherwise}
\end{cases}.
\end{equation}
\end{enumerate}
\subsection{Edge State Prediction}
We wish to compute a left and right state of primitive variables at each edge to be used as inputs to the Riemann problem.  We use a version of the Colella and Sekora 2009 PPM algorithm, which has been further modified to eliminate sensitivity due to roundoff error (modifications via personal communication with Colella).  Note that CASTRO also has options for the original PPM algorithm of Colella and Woodward 1984, and piecewise-linear algorithm described in Saltzman 1994.  We also use characteristic tracing with corner coupling in 3D, as described in Miller and Colella 2002.  We give full details of the PPM algorithm, as it has not appeared before in the literature, and summarize the developments from Miller and Colella 2002.

The PPM algorithm is used to compute time-centered edge states by extrapolating the base-time data in space and time.  The edge states are dual-valued, i.e., at each face, there is a left state and a right state estimate.  The spatial extrapolation is one-dimensional, i.e., transverse derivatives are ignored.  We also use a flattening procedure to further limit the edge state values.  The Miller and Colella 2002 algorithm, which we describe later, incorporates the transverse terms, and also describes the modifications required for equations with additional characteristics besides the fluid velocity.  There are four steps to compute these dual-valued edge states (here, we use $s$ to denote an arbitrary scalar from $\Qb$, and we write the equations in 1D, for simplicity):
\begin{itemize}
\item {\bf Step 1}: Compute $s_{i,+}$ and $s_{i,-}$, which are spatial interpolations of $s$ to the hi and lo side of the face with special limiters, respectively.  Begin by interpolating $s$ to edges using a 4th-order interpolation in space:
\begin{equation}
s_{i+\myhalf} = \frac{7}{12}\left(s_{i+1}+s_i\right) - \frac{1}{12}\left(s_{i+2}+s_{i-1}\right).
\end{equation}
Then, if $(s_{i+\myhalf}-s_i)(s_{i+1}-s_{i+\myhalf}) < 0$, we limit $s_{i+\myhalf}$ 
a nonlinear combination of approximations to the second derivative.  The steps are as follows:
\begin{enumerate}
\item Define:
\begin{eqnarray}
(D^2s)_{i+\myhalf} &=& 3\left(s_{i}-2s_{i+\myhalf}+s_{i+1}\right) \\
(D^2s)_{i+\myhalf,L} &=& s_{i-1}-2s_{i}+s_{i+1} \\
(D^2s)_{i+\myhalf,R} &=& s_{i}-2s_{i+1}+s_{i+2}
\end{eqnarray}
\item Define
\begin{equation}
s = \text{sign}\left[(D^2s)_{i+\myhalf}\right],
\end{equation}
\begin{equation}
(D^2s)_{i+\myhalf,\text{lim}} = s\max\left\{\min\left[Cs\left|(D^2s)_{i+\myhalf,L}\right|,Cs\left|(D^2s)_{i+\myhalf,R}\right|,s\left|(D^2s)_{i+\myhalf}\right|\right],0\right\},
\end{equation}
where $C=1.25$ as used in Colella and Sekora 2009.  The limited value of $s_{i+\myhalf}$ is
\begin{equation}
s_{i+\myhalf} = \frac{1}{2}\left(s_{i}+s_{i+1}\right) - \frac{1}{6}(D^2s)_{i+\myhalf,\text{lim}}.
\end{equation}
\end{enumerate}
Now we implement an updated implementation of the Colella and Sekora 2009 algorithm which eliminates sensitivity to roundoff.  First we need to detect whether a particular cell corresponds to an ``extremum''.  There are two tests.
\begin{itemize}
\item For the first test, define
\begin{equation}
\alpha_{i,\pm} = s_{i\pm\myhalf} - s_i.
\end{equation}
If $\alpha_{i,+}\alpha_{i,-} \ge 0$, then we are at an extremum.
\item We only apply the second test if either $|\alpha_{i,\pm}| > 2|\alpha_{i,\mp}|$.  
If so, we define:
\begin{eqnarray}
(Ds)_{i,{\rm face},-} &=& s_{i-\myhalf} - s_{i-\sfrac{3}{2}} \\
(Ds)_{i,{\rm face},+} &=& s_{i+\sfrac{3}{2}} - s_{i-\myhalf}
\end{eqnarray}
\begin{equation}
(Ds)_{i,{\rm face,min}} = \min\left[\left|(Ds)_{i,{\rm face},-}\right|,\left|(Ds)_{i,{\rm face},+}\right|\right].
\end{equation}
\begin{eqnarray}
(Ds)_{i,{\rm cc},-} &=& s_{i} - s_{i-1} \\
(Ds)_{i,{\rm cc},+} &=& s_{i+1} - s_{i}
\end{eqnarray}
\begin{equation}
(Ds)_{i,{\rm cc,min}} = \min\left[\left|(Ds)_{i,{\rm cc},-}\right|,\left|(Ds)_{i,{\rm cc},+}\right|\right].
\end{equation}
If $(Ds)_{i,{\rm face,min}} \ge (Ds)_{i,{\rm cc,min}}$, set 
$(Ds)_{i,\pm} = (Ds)_{i,{\rm face},\pm}$.  Otherwise, set 
$(Ds)_{i,\pm} = (Ds)_{i,{\rm cc},\pm}$.  Finally, we are at an extreumum if
$(Ds)_{i,+}(Ds)_{i,-} \le 0$.
\end{itemize}
Thus concludes the extremum tests.  The remaining limiters depend on whether we are at an extremum.
\begin{itemize}
\item If we are at an extremum, we modify $\alpha_{i,\pm}$.  First, we define
\begin{eqnarray}
(D^2s)_{i} &=& 6(\alpha_{i,+}+\alpha_{i,-}) \\
(D^2s)_{i,L} &=& s_{i-2}-2s_{i-1}+s_{i} \\
(D^2s)_{i,R} &=& s_{i}-2s_{i+1}+s_{i+2} \\
(D^2s)_{i,C} &=& s_{i-1}-2s_{i}+s_{i+1}
\end{eqnarray}
Then, define
\begin{equation}
s = \text{sign}\left[(D^2s)_{i}\right],
\end{equation}
\begin{equation}
(D^2s)_{i,\text{lim}} = \max\left\{\min\left[s(D^2s)_{i},Cs\left|(D^2s)_{i,L}\right|,Cs\left|(D^2s)_{i,R}\right|,Cs\left|(D^2s)_{i,C}\right|\right],0\right\}.
\end{equation}
Then,
\begin{equation}
\alpha_{i,\pm} = \frac{\alpha_{i,\pm}(D^2s)_{i,\text{lim}}}{\max\left[(D^2s)_{i},1\times 10^{-10}\right]}
\end{equation}
\item If we are not at an extremum and $|\alpha_{i,\pm}| > 2|\alpha_{i,\mp}|$, then define
\begin{equation}
s = \text{sign}(\alpha_{i,\mp})
\end{equation}
\begin{equation}
\delta\mathcal{I}_{\text{ext}} = \frac{-\alpha_{i,\pm}^2}{4\left(\alpha_{j,+}+\alpha_{j,-}\right)},
\end{equation}
\begin{equation}
\delta s = s_{i\mp 1} - s_i,
\end{equation}
If $s\delta\mathcal{I}_{\text{ext}} \ge s\delta s$, then we perform the following test.
If $s\delta s - \alpha_{i,\mp} \ge 1\times 10^{-10}$, then
\begin{equation}
\alpha_{i,\pm} =  -2\delta s - 2s\left[(\delta s)^2 - \delta s \alpha_{i,\mp}\right]^{\myhalf}
\end{equation}
otherwise,
\begin{equation}
\alpha_{i,\pm} =  -2\alpha_{i,\mp}
\end{equation}
\end{itemize}
Finally, $s_{i,\pm} = s_i + \alpha_{i,\pm}$.
\item {\bf Step 2}: Construct a quadratic profile using $s_{i,-},s_i$, and $s_{i,+}$.
\begin{equation}
s_i^I(x) = s_{i,-} + \xi\left[s_{i,+} - s_{i,-} + s_{6,i}(1-\xi)\right],\label{Quadratic Interp}
\end{equation}
\begin{equation}
s_6 = 6s_{i} - 3\left(s_{i,-}+s_{i,+}\right),
\end{equation}
\begin{equation}
\xi = \frac{x - ih}{h}, ~ 0 \le \xi \le 1.
\end{equation}
\item {\bf Step 3:} Integrate quadratic profiles.  We are essentially computing the average value swept out by the quadratic profile across the face assuming the profile is moving at a speed $\lambda_k$.\\ \\
Define the following integrals, where $\sigma_k = |\lambda_k|\Delta t/h$:
\begin{eqnarray}
\mathcal{I}_{i,+}(\sigma_k) &=& \frac{1}{\sigma_k h}\int_{(i+\myhalf)h-\sigma_k h}^{(i+\myhalf)h}s_i^I(x)dx \\
\mathcal{I}_{i,-}(\sigma_k) &=& \frac{1}{\sigma_k h}\int_{(i-\myhalf)h}^{(i-\myhalf)h+\sigma_k h}s_i^I(x)dx
\end{eqnarray}
Plugging in (\ref{Quadratic Interp}) gives:
\begin{eqnarray}
\mathcal{I}_{i,+}(\sigma_k) &=& s_{i,+} - \frac{\sigma_k}{2}\left[s_{i,+}-s_{i,-}-\left(1-\frac{2}{3}\sigma_k\right)s_{6,i}\right], \\
\mathcal{I}_{i,-}(\sigma_k) &=& s_{i,-} + \frac{\sigma_k}{2}\left[s_{i,+}-s_{i,-}+\left(1-\frac{2}{3}\sigma_k\right)s_{6,i}\right].
\end{eqnarray}
\item {\bf Step 4:} Obtain 1D edge states by performing a 1D extrapolation to get 
left and right edge states.  Note that we include an explicit source term contribution.
\begin{eqnarray}
s_{L,i+\myhalf} &=& s_i - \chi_i\sum_{k:\lambda_k \ge 0}\lb_k\cdot\left[s_i-\mathcal{I}_{i,+}(\sigma_k)\right]\rb_k + \frac{\dt}{2}S_i^n, \\
s_{R,i-\myhalf} &=& s_i - \chi_i\sum_{k:\lambda_k < 0}\lb_k\cdot\left[s_i-\mathcal{I}_{i,-}(\sigma_k)\right]\rb_k + \frac{\dt}{2}S_i^n.
\end{eqnarray}
Here, $\rb_k$ is the $k^{\rm th}$ right column eigenvector of $\Rb(\Ab_d)$ and $\lb_k$ is the $k^{\rm th}$ left row eigenvector lf $\Lb(\Ab_d)$.  The flattening coefficient is $\chi_i$.
\end{itemize}
In order to add the transverse terms in an spatial operator unsplit framework, the details follow exactly as given in Section 4.2.1 in Miller and Colella 2002, except for the details of the Riemann solver, which are given below.
\subsection{Riemann Problem}
Inputs from the edge state prediction are $\rho_{L/R}, u_{L/R}, v_{L/R}, p_{L/R}$, and $(\rho e)_{L/R}$ ($v$ represents all of the transverse velocity components).  We also compute $\gamma$ at cell centers and copy these to edges directly to get the left and right states, $\gamma_{L/R}$.  We also define $c_{\rm avg}$ as a face-centered value that is the average of the neighboring cell-centered values of $c$.  We have also computed $\rho_{\rm small}, p_{\rm small}$, and $c_{\rm small}$ using cell-centered data.  

Here are the steps.  First, define $(\rho c)_{\rm small} = \rho_{\rm small}c_{\rm small}$. Then, define:
\begin{equation}
(\rho c)_{L/R} = \max\left[(\rho c)_{\rm small},\left|\gamma_{L/R},p_{L/R},\rho_{L/R}\right|\right].
\end{equation}
Define star states:
\begin{equation}
p^* = \max\left[p_{\rm small},\frac{\left[(\rho c)_L p_R + (\rho c)_R p_L\right] + (\rho c)_L(\rho c)_R(u_L-u_R)}{(\rho c)_L + (\rho c)_R}\right],
\end{equation}
\begin{equation}
u^* = \frac{\left[(\rho c)_L u_L + (\rho c)_R u_R\right]+ (p_L - p_R)}{(\rho c)_L + (\rho c)_R}.
\end{equation}
If $u^* \ge 0$ then define $\rho_0, u_0, p_0, (\rho e)_0$ and $\gamma_0$ to be the left state.  Otherwise, define them to be the right state.  Then, set
\begin{equation}
\rho_0 = \max(\rho_{\rm small},\rho_0),
\end{equation}
and define
\begin{equation}
c_0 = \max\left(c_{\rm small},\sqrt{\frac{\gamma_0 p_0}{\rho_0}}\right),
\end{equation}
\begin{equation}
\rho^* = \rho_0 + \frac{p^* - p_0}{c_0^2},
\end{equation}
\begin{equation}
(\rho e)^* = (\rho e)_0 + (p^* - p_0)\frac{(\rho e)_0 + p_0}{\rho_0 c_0^2},
\end{equation}
\begin{equation}
c^* = \max\left(c_{\rm small},\sqrt{\left|\frac{\gamma_0 p^*}{\rho^*}\right|}\right)
\end{equation}
Then,
\begin{eqnarray}
c_{\rm out} &=& c_0 - {\rm sign}(u^*)u_0, \\
c_{\rm in} &=& c^* - {\rm sign}(u^*)u^*, \\
c_{\rm shock} &=& \frac{c_{\rm in} + c_{\rm out}}{2}.
\end{eqnarray}
If $p^* - p_0 \ge 0$, then $c_{\rm in} = c_{\rm out} = c_{\rm shock}$.  Then, if $c_{\rm out} = c_{\rm in}$, we define $c_{\rm temp} = \epsilon c_{\rm avg}$.  Otherwise, $c_{\rm temp} = c_{\rm out} - c_{\rm in}$.  We define the fraction
\begin{equation}
f = \half\left[1 + \frac{c_{\rm out} + c_{\rm in}}{c_{\rm temp}}\right],
\end{equation}
and constrain $f$ to lie in the range $f\in[0,1]$.

To get the final ``Godunov'' state, for the transverse velocity, we upwind based on $u^*$.
\begin{equation}
v_{\rm gdnv} =
\begin{cases}
v_L, & u^* \ge 0 \\
v_R, & {\rm otherwise}
\end{cases}.
\end{equation}
Then, define
\begin{eqnarray}
\rho_{\rm gdnv} &=& f\rho^* + (1-f)\rho_0, \\
u_{\rm gdnv} &=& f u^* + (1-f)u_0, \\
p_{\rm gdnv} &=& f p^* + (1-f)p_0, \\
(\rho e)_{\rm gdnv} &=& f(\rho e)^* + (1-f)(\rho e)_0.
\end{eqnarray}
Finally, if $c_{\rm out} < 0$, set $\rho_{\rm gdnv}=\rho_0, u_{\rm gdnv}=u_0, p_{\rm gdnv}=p_0$, and $(\rho e)_{\rm gdnv}=(\rho e)_0$.  If $c_{\rm in}\ge 0$, set $\rho_{\rm gdnv}=\rho^*, u_{\rm gdnv}=u^*, p_{\rm gdnv}=p^*$, and $(\rho e)_{\rm gdnv}=(\rho e)^*$.
\subsection{Compute Fluxes and Update}
Compute the fluxes as a function of the primitive variables, and then advance the solution:
\begin{equation}
\Ub^{n+1} = \Ub^n - \dt\nabla\cdot\Fb^\nph + \dt\Sb^n.
\end{equation}
Again, note that since the source term is not time centered, this is not a second-order method.  After the advective update, we correct the solution, effectively time-centering the source term.
