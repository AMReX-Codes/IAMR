
\section{Getting Started With CVS}

CASTRO is distributed through CVS access to the CASTRO repository, which is stored at LBNL.
\begin{enumerate}
\item If you would like to use CASTRO, the first thing you have to do is contact John Bell (jbbell@lbl.gov). If you are a sanctioned user, he will set you up with a user account on battra.lbl.gov. You will need to make phone contact with him once the account is set up in order to get the temporary password.
\item Once you have an account on battra@lbl.gov, please change your password as soon as you log on. You will then need to set the following variables in your account:\\ \\
CVS\_RSH \\
CVSROOT \\ \\
John will email you that information once your account is set up. This should be the only time you actually need to log in to battra directly. However, the password you set is the one you use every time you checkout new code from battra, or "cvs update" the code you already have.
\item Once 1) and 2) are complete, type (on your machine)

"cvs co CASTRO"

"co" here is a shorthand for "checkout"

"CASTRO" is the name of a CVS module, which we have defined to include all the directories and subdirectories that you will need to successfully build and run CASTRO.

You will need to enter your battra password here.

\end{enumerate}

\section{Building the Code}

\begin{enumerate}

\item From the directory in which you checked out CASTRO, type

cd CASTRO/Parallel/Castro/Sedov

This will put you into a "run" directory in which you can run the Sedov problem in 1-d, 2-d or 3-d.
\item In Sedov, edit the GNUmakefile, and set

DIM = 2 (for example)

COMP = your favorite C++ compiler

FCOMP = your favorite Fortran compiler (which must compile F90)

DEBUG = FALSE

We like COMP = Intel and FCOMP = Intel. Some users successfully use 
COMP = gcc and FCOMP = gfortran. If you want to try other compilers and they don't work, 
please let us know.  (Note from Andy: g++ version 4.4 didn't work with DIM=1.  I had to revert to version 4.1.  gfortran version 4.4 works fine)

\item Now type "make". The resulting executable will look something like 
"Castro2d.Linux.Intel.Intel.ex", which means this is a 2-d version of the code, 
made on a Linux machine, with COMP = Intel and FCOMP = Intel.

\end{enumerate}

\section{Running the Code}

\begin{enumerate}

\item Type "Castro2d.Linux.Intel.Intel.ex inputs.2d.cyl\_in\_cartcoords" 
This will run the 2-d cylindrical Sedov problem in Cartesian (x-y coordinates). 
You can see other possible options, which should be clear by the names of the inputs files.

\item You will notice that running the code generates directories that look like 
plt00000, plt00020, etc, and chk00000, chk00020, etc. These are "plotfiles" and 
"checkpoint" files. The plotfiles are used for visualization, the checkpoint files are 
used for restarting the code.

\end{enumerate}

\section{Visualization of the Results}

\begin{enumerate}

\item To visualize the plotfiles, you will need to go into the CASTRO/Parallel/pAmrvis 
directory. Edit the GNUmakefile there to set DIM = 2, and again set COMP and FCOMP to 
compilers that you have. Leave DEBUG = FALSE. Then type "make".  This will make an 
executable that looks like "amrvis2d...ex".

[Note from Andy] If you want to build amrvis with DIM = 3, you must first build volpack by 
typing ``make'' in CASTRO/Parallel/volpack.

[Note from Ann] This requires the OSF/Motif libraries and headers, if you don't have these 
you will need to install the development version of motif through your package manager. 
lesstif gives some functionality and will allow you to build the amrvis executable, 
but amrvis will not run properly.

[Note from Shaw] On most Linux distributions, motif library is provided by the openmotif package, 
and its header files (like Xm.h) are provided by openmotif-devel. If those packages are not 
installed, then use the package management tool to install them, which varies from distribution 
to distribution, but is straightforward. I can provide detailed instructions if anyone needs them.

You may then want to create an alias to amrvis2d, for example

alias amrvis2d /work/CASTRO/Parallel/pAmrvis/amrvis2d...ex

\item Return to the CASTRO/Parallel/Castro/Sedov directory. Type "amrvis2d plt00152" to see a single plotfile, or "amrvis2d -a plt*", which will animate the sequence of plotfiles. Try playing around with this -- note you can change which variable you are looking at, you can select a region and click "Dataset" (under View) in order to look at the actual numbers, etc. You can also export the pictures in several different formats -- under "File", see "Export".

Please know that we do have a number of conversion routines to other formats (such as matlab), but it is hard to describe them all. If you would like to display the data in another format, please let us know (again, asalmgren@lbl.gov) and we will point you to whatever we have that can help.

\end{enumerate}

You have now completed a brief introduction to CASTRO. 
